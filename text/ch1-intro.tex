\chapter{\label{ch:1-intro}Introduction} 

\minitoc

\section{Alzheimer's Disease}
\section{Mouse Models of AD: APP$^{NL-G-F}$}
% TODO: Something more introductary, e.g. about transgenic mice?
% TODO: then emphasising how it is better?
The APP$^{NL-G-F}$ mouse model is based on a C57BL/6 background carrying mutations of human familial AD patients, namely the Swedish (NL), Beyreuther/Iberian (F) and Arctic (G) mutations.
This model of Aβ amyloidosis notably does not show any overexpression of amyloid precursor protein (APP), but rather overproduction of A\textbeta$_{42}$. Thus, these mice exhibit Aβ amyloidosis, with plaque deposition in the cortex from as early as 2 months of age, showing near saturation at 7 months of age. Notably, there is considerable microgliosis and astrocytosis as hallmarks of neuroinflammation. Furthermore,  APP$^{NL-G-F}$ exhibit moderate memory impairments at 6 months of age, as measured by reduced alternation in the Y-Maze test. The age-dependent nature of pathology in this model is better suited to model the course of the disease in humans. \citep{Saito2014}
\section{Recording Activity of Neurons and Oligodendrocytes: In Vivo Two-Photon Calcium Imaging}
% TODO: some citation needed, more explanation about calcium indicators, two-photon effect
Neural activity can be detected using calcium indicators: During neuronal activity, extracellular calcium enters the cell.\\
Among calcium indicators, the subclass of genetically encoded calcium indicators (GECI) are especially frequently used. An optimal GECI exhibits high levels of expression in desired cells, high sensitivity (detection of action potentials), has fast kinetics (short rise time, short decay time), a high signal-to-noise ratio, a high tolerance to saturation, a high temporal resolution and a linear increase of fluorescence when spiking events coincide. Also, especially for recordings weeks after the injection, GECIs should maintain their fluorescence dynamics over time, should have little to no cytotoxic properties and should not suffer extensively from photobleaching. \cite{Zhang2023} \\
\textit{Zhang et al.} designed a family of calcium sensors, GCaMP8, a GECI based on green fluorescent protein (GFP). This calcium sensor is coupled to a synapsin-1 promotor, resulting in broad expression in neurons, excluding the nuclei. The entities in this family show different properties: GCaMP8f (\textit{fast}) exhibits especially fast rise and decay times, but lower sensitivity. GCaMP8s (\textit{sensitive}), on the other hand, has higher sensitivity, but slower kinetics as a trade-off. GCaMP8m (\textit{medium}) is a compromise with fast rise but medium decay times. \cite{Zhang2023} \\
This project required a calcium sensor that is especially sensitive and able to resolve single action potentials up to trains of spikes. Also, as the time scale of this project span several months, maintaining the same calcium dynamics over time was crucial. \\
For this project, we chose to use GCaMP8 because of its robust sensitivity and favourable dynamics. In addition, this virus allows for long-scale imaging projects as dynamics are almost unchanged over short periods of time, even weeks after the injection.  \cite{Zhang2023} We then decided to use the GCaMP8\textit{s} subtype because of its higher sensitivity and still more than robust temporal resolution. 
\section{Dysfunction of Oligodendrocytes in Alzheimer’s Disease}
% TODO: Sasmita et al.??
% TODO: add that oligodendrocytes produce even more Aβ?
Using single-nucleus RNA sequencing in human brain tissue of AD patients and healthy controls, previous work could show that oligodendrocytes express all the genes required to produce Aβ, including amyloid precursor protein (APP), beta secretase (BACE1) and components of gamma secretase such as presenilin 1 (PSEN1). Aβ production has been shown in oligodendrocytes derived from human familial AD patients' induced pluripotent stem cells (iPSCs). By inhibiting BACE1 through the administration of NB-360 in said oligodendrocytes, Aβ production could be significantly reduced to approximately 20\% of pre-treatment levels. An APP$^{NL-G-F}$ mouse model which allowed for conditional BACE1 knockout in either oligodendrocytes or neurons has been generated. When conditionally knocking out BACE1 in oligodendrocytes of 4 month old mice, the plaque load measured in number of plaques per mm$^2$ was reduced by 25\% compared to unmodified APP$^{NL-G-F}$ mice. Interestingly, in APP$^{NL-G-F}$ mice where BACE1 was knocked out in neurons, plaques were almost eliminated. Furthermore, neuronal hyperactivity in APP$^{NL-G-F}$ mice assessed by mean firing rate in hertz could be ameliorated by knocking out BACE1, both in neurons and oligodendrocytes. \citep{Rajani2024}
\break
Another study knocking out BACE1 in 6 month old mice with the same genetic background (APP$^{NL-G-F}$) in either oligodendrocytes or excitatory neurons further validated these findings as they reported a 30\% and 95-98\% reduction of plaque burden compared to controls, respectively. Interestingly, there was still noticeable plaque deposition in brains of ExN-BACE1 knockout mice at 12 months of age, further underlining the existence of non-neuronal sources of Aβ.\citep{Sasmita2024}
% TODO: there are more papers showing the latter argument!
% TODO: talk about 'sigmoidal growth curve', and threshold levels?
% TODO: Careful: the last sentence is prety close to sasmita et al.
Both studies highlight that plaque deposition depends on neuronal Aβ and the non-linear relation of APP processing, Aβ production and plaque formation. \citep{Sasmita2024, Rajani2024}