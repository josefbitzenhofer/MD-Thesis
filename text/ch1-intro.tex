\chapter{\label{ch:1-intro}Introduction} 

\minitoc

\section{Alzheimer's Disease}
\section{Mouse Models of AD: APP$^{NL-G-F}$}
% TODO: Something more introductary, e.g. about transgenic mice?
% TODO: then emphasising how it is better?
The APP$^{NL-G-F}$ mouse model is based on a C57BL/6 background carrying mutations of human familial AD patients, namely the Swedish (NL), Beyreuther/Iberian (F) and Arctic (G) mutations.
This model of Aβ amyloidosis notably does not show any overexpression of amyloid precursor protein (APP), but rather overproduction of A\textbeta$_{42}$. Thus, these mice exhibit Aβ amyloidosis, with plaque deposition in the cortex from as early as 2 months of age, showing near saturation at 7 months of age. Notably, there is considerable microgliosis and astrocytosis as hallmarks of neuroinflammation. Furthermore,  APP$^{NL-G-F}$ exhibit moderate memory impairments at 6 months of age, as measured by reduced alternation in the Y-Maze test. The age-dependent nature of pathology in this model is better suited to model the course of the disease in humans. \citep{Saito2014}
\section{Recording Activity of Neurons and Oligodendrocytes: In-Vivo Two-Photon Calcium Imaging}
% TODO: cite Grienberger et al. in any capacity???
% TODO: reviews, sources, etc. and then enhance pls!
Over recent decades, much progress has been made to record neural activity in-vitro and in-vivo. Classic electrophysiological methods include patch-clamping or probe recordings (TODO: is that right?) and more recently Neuropixels, a new probe which allows for large scale recordings of neurons in the mouse brain. \\
Furthermore, neuroscientific studies progressively aim at putting cellular activity into a larger both spatial and temporal context, thus allowing investigation of whole networks of cells. \cite{Denk2005}\\
In addition to that, in-vivo approaches in neuroscience strive for correlating activity of individual cells to entire networks of cells with the animal's behaviour. \cite{Denk2005}
% TODO: add some example studies for animal behaviour neuro pattern correlations
% TODO: citation needed for all of it and phrasing is catastrophic
% TODO: should I cite Denk Helmchen 2005 for "optophysiology"?
However, efforts have been made to use fluorescence imaging to detect neural activity, being an optophysiological rather than an electrophysiological approach. 
\subsection{Fluorescence Imaging}
% TODO: talk ablout linear fluorescence imaging and find citations!
When performing linear fluorescence imaging (one-photon imaging) such as confocal imaging, the depth from surface of the tissue is limited as scattering of light rays would occur. \cite{Denk2005} \\
% TODO: Probably cite one of the reviews here?
Nowadays, two-photon calcium imaging (2PCI) is a standard method, widely accepted to investigate activity of large scale neural networks. \\
2PCI overcomes some of the limitations in linear fluorescence imaging and has some advantages over \textit{classic} electrophysiological approaches. In contrast to one-photon methods, 2PCI allows for rather stable resolution and contrast, even in scattering tissue. In addition, 2PCI enables neuroscientists to study cells in their natural environment and their responses to manipulations. \cite{Denk2005} The implantation of chronic glass cranial windows allows for investigation of cell structure and activity within the same animal over time. \cite{Denk2005}
% TODO: add references for chronic windows from Denk, Helmchen 2005 and/or add some example studies here
\subsection{Two-photon Effect}
In two-photon microscopy, a molecular dye, such as a calcium indicator, can be excited when it absorbs two photons at the same time.\cite{Denk1990} 
% TODO: will I have to talk about excitation and emission wavelengths here? also the distance in between the two photons in time?
\subsection{Two-photon Microscope}
A two-photon microscope consists of multiple components to optimise its optical pathway for imaging of biological tissue: A pulsed titanium sapphire laser (short: Ti:sapphire) is generating short pulses (<100 fs) at 100 MHz, and can be tuned to a wavelength of 670-1070 nm. The beam is then expanded using a telescope. \cite{Denk2005} A half-wave plate is combined with either a polarising cube or a Pockels cell to allow for laser power modulation. Additionally, exposure of the laser beam can be stopped by closing a mechanical shutter. Commonly, beam scanning systems either use galvo-galvo or galvo-resonant mirrors.  Lightweight galvanometer mounted mirrors (short: galvos) can be rotated orthogonally to the xy-axis ("xy-deflection module" \cite{Denk2005}). \cite{Grienberger2022} Scan lenses and tube lenses are used to further expand the laser beam. \cite{Denk2005} Furthermore, the optical pathway is directed either from laser to sample or from sample to photodetection by a dichroic mirror. The beam passes the objective lens to reach the sample. The signal from the sample is passed through a collection lens and then collected by photomultiplier tubes (PMTs), most commonly gallium arsenide (GAsP) PMTs. \cite{Grienberger2022}
\\
% TODO: Grienberger et al. 2022 give example studies for this, but I need to follow them up or find my own examples??
% TODO: the first sentence is very similar to Grienberger et al.
When using a resonant-galvo scanning system, it is possible to achieve frame rates and resolutions which allow for detection of neuronal activity patterns on the timescale of behaviour. Thus, raster scanning with a combination of resonant and galvo scanners has become the primary method in 2PCI. \cite{Grienberger2022} \\
%  TODO: Grienberger et al. cite the protocls, can I do that as well?
In order to gain optical access to the brain, two commonly used methods are open-skull windows or a thin-skull preparation. \cite{Grienberger2022} However, the open-skull method is preferred as it allows for chronic recordings \cite{Grienberger2022}, \cite{Denk2005} a larger area to be imaged from and provides higher resolution, especially when imaging from deeper structures. \cite{Grienberger2022} 
    % \item in theory, you can have a large enough implant to image from the entire cortex → check reference \textbf{51, 53} + find more examples??
    % \item simultaneous ephys: partial windows / silicon-based polymer films → check reference \textbf{54}
    % \item microprisms → PFC, multiple cortical layers. → check references \textbf{55, 56, 57}
    % \item glass cannula plugs → hippocampus, striatum → check references \textbf{58, 59, 60}
    % \item GRIN lenses → \textbf{62, 63, 64}
2PCI in conjunction with chronic glass windows allow for versatile combinations, thus expanding the brain regions and timescales achievable:\\
By increasing the diameter of the cranial window, imaging the entire cortex is possible. citation needed! Using partial cranial windows or by replacing glass with silicon-based polymer films, simultaneous electrophysiological recordings and 2PCI can be performed. citation needed! Furthermore, the implantation of microprisms allows for imaging of structures like the prefrontal cortex or even multiple layers in the cortex. citation needed! By implanting glass cannula plugs, deeper structures such as the hippocampus or striatum can be imaged from. citation needed! Lastly, the invention of GRIN lenses....
\subsection{Calcium indicators}
% TODO: some citation needed, more explanation about calcium indicators, two-photon effect
Neural activity can be detected using calcium indicators: During neuronal activity, extracellular calcium enters the cell.\\
Among calcium indicators, the subclass of genetically encoded calcium indicators (GECI) are especially frequently used. An optimal GECI exhibits high levels of expression in desired cells, high sensitivity (detection of action potentials), has fast kinetics (short rise time, short decay time), a high signal-to-noise ratio, a high tolerance to saturation, a high temporal resolution and a linear increase of fluorescence when spiking events coincide. Also, especially for recordings weeks after the injection, GECIs should maintain their fluorescence dynamics over time, should have little to no cytotoxic properties and should not suffer extensively from photobleaching. \cite{Zhang2023} \\
\textit{Zhang et al.} designed a family of calcium sensors, GCaMP8, a GECI based on green fluorescent protein (GFP). This calcium sensor is coupled to a synapsin-1 promotor, resulting in broad expression in neurons, excluding the nuclei. The entities in this family show different properties: GCaMP8f (\textit{fast}) exhibits especially fast rise and decay times, but lower sensitivity. GCaMP8s (\textit{sensitive}), on the other hand, has higher sensitivity, but slower kinetics as a trade-off. GCaMP8m (\textit{medium}) is a compromise with fast rise but medium decay times. \cite{Zhang2023} \\
This project required a calcium sensor that is especially sensitive and able to resolve single action potentials up to trains of spikes. Also, as the time scale of this project span several months, maintaining the same calcium dynamics over time was crucial. \\
For this project, we chose to use GCaMP8 because of its robust sensitivity and favourable dynamics. In addition, this virus allows for long-scale imaging projects as dynamics are almost unchanged over short periods of time, even weeks after the injection.  \cite{Zhang2023} We then decided to use the GCaMP8\textit{s} subtype because of its higher sensitivity and still more than robust temporal resolution.
\subsection{From Calcium Traces to Neural Activity}
% Bad phrasing, but talk about "validation", "deconvolution" etc.
% TODO: some citation needed, more explanation about calcium indicators, two-photon effect
Neural activity can be detected using calcium indicators: During neuronal activity, extracellular calcium enters the cell.\\
Among calcium indicators, the subclass of genetically encoded calcium indicators (GECI) are especially frequently used. An optimal GECI exhibits high levels of expression in desired cells, high sensitivity (detection of action potentials), has fast kinetics (short rise time, short decay time), a high signal-to-noise ratio, a high tolerance to saturation, a high temporal resolution and a linear increase of fluorescence when spiking events coincide. Also, especially for recordings weeks after the injection, GECIs should maintain their fluorescence dynamics over time, should have little to no cytotoxic properties and should not suffer extensively from photobleaching. \cite{Zhang2023} \\
\textit{Zhang et al.} designed a family of calcium sensors, GCaMP8, a GECI based on green fluorescent protein (GFP). This calcium sensor is coupled to a synapsin-1 promotor, resulting in broad expression in neurons, excluding the nuclei. The entities in this family show different properties: GCaMP8f (\textit{fast}) exhibits especially fast rise and decay times, but lower sensitivity. GCaMP8s (\textit{sensitive}), on the other hand, has higher sensitivity, but slower kinetics as a trade-off. GCaMP8m (\textit{medium}) is a compromise with fast rise but medium decay times. \cite{Zhang2023} \\
This project required a calcium sensor that is especially sensitive and able to resolve single action potentials up to trains of spikes. Also, as the time scale of this project span several months, maintaining the same calcium dynamics over time was crucial. \\
For this project, we chose to use GCaMP8 because of its robust sensitivity and favourable dynamics. In addition, this virus allows for long-scale imaging projects as dynamics are almost unchanged over short periods of time, even weeks after the injection.  \cite{Zhang2023} We then decided to use the GCaMP8\textit{s} subtype because of its higher sensitivity and still more than robust temporal resolution. 
\section{Dysfunction of Oligodendrocytes in Alzheimer’s Disease}
% TODO: Sasmita et al.??
% TODO: add that oligodendrocytes produce even more Aβ?
Using single-nucleus RNA sequencing in human brain tissue of AD patients and healthy controls, previous work could show that oligodendrocytes express all the genes required to produce Aβ, including amyloid precursor protein (APP), beta secretase (BACE1) and components of gamma secretase such as presenilin 1 (PSEN1). Aβ production has been shown in oligodendrocytes derived from human familial AD patients' induced pluripotent stem cells (iPSCs). By inhibiting BACE1 through the administration of NB-360 in said oligodendrocytes, Aβ production could be significantly reduced to approximately 20\% of pre-treatment levels. An APP$^{NL-G-F}$ mouse model which allowed for conditional BACE1 knockout in either oligodendrocytes or neurons has been generated. When conditionally knocking out BACE1 in oligodendrocytes of 4 month old mice, the plaque load measured in number of plaques per mm$^2$ was reduced by 25\% compared to unmodified APP$^{NL-G-F}$ mice. Interestingly, in APP$^{NL-G-F}$ mice where BACE1 was knocked out in neurons, plaques were almost eliminated. Furthermore, neuronal hyperactivity in APP$^{NL-G-F}$ mice assessed by mean firing rate in hertz could be ameliorated by knocking out BACE1, both in neurons and oligodendrocytes. \parencite{Rajani2024}
\break
Another study knocking out BACE1 in 6 month old mice with the same genetic background (APP$^{NL-G-F}$) in either oligodendrocytes or excitatory neurons further validated these findings as they reported a 30\% and 95-98\% reduction of plaque burden compared to controls, respectively. Interestingly, there was still noticeable plaque deposition in brains of ExN-BACE1 knockout mice at 12 months of age, further underlining the existence of non-neuronal sources of Aβ.\citep{Sasmita2024}
% TODO: there are more papers showing the latter argument!
% TODO: talk about 'sigmoidal growth curve', and threshold levels?
% TODO: Careful: the last sentence is prety close to sasmita et al.
Both studies highlight that plaque deposition depends on neuronal Aβ and the non-linear relation of APP processing, Aβ production and plaque formation. \citep{Sasmita2024, Rajani2024}