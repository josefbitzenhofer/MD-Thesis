%%%%%%%%%%%%%%%%%%%%%%%%%%%%%%%%%%%%%%%%%%%%%%%%%%%%%%%%%%%%%%%
%% OXFORD THESIS TEMPLATE

% Use this template to produce a standard thesis that meets the Oxford University requirements for DPhil submission
%
% Originally by Keith A. Gillow (gillow@maths.ox.ac.uk), 1997
% Modified by Sam Evans (sam@samuelevansresearch.org), 2007
% Modified by John McManigle (john@oxfordechoes.com), 2015
%
% This version Copyright (c) 2015-2023 John McManigle
%
% Broad permissions are granted to use, modify, and distribute this software
% as specified in the MIT License included in this distribution's LICENSE file.
%

% I've (John) tried to comment this file extensively, so read through it to see how to use the various options.  Remember
% that in LaTeX, any line starting with a % is NOT executed.  Several places below, you have a choice of which line to use
% out of multiple options (eg draft vs final, for PDF vs for binding, etc.)  When you pick one, add a % to the beginning of
% the lines you don't want.


%%%%% CHOOSE PAGE LAYOUT
% The most common choices should be below.  You can also do other things, like replacing "a4paper" with "letterpaper", etc.

% This one will format for two-sided binding (ie left and right pages have mirror margins; blank pages inserted where needed):
\documentclass[a4paper,twoside]{ociamthesis}
% This one will format for one-sided binding (ie left margin > right margin; no extra blank pages):
%\documentclass[a4paper]{ociamthesis}
% This one will format for PDF output (ie equal margins, no extra blank pages):
%\documentclass[a4paper,nobind]{ociamthesis} 



%%%%% SELECT YOUR DRAFT OPTIONS
% Three options going on here; use in any combination.  But remember to turn the first two off before
% generating a PDF to send to the printer!

% This adds a "DRAFT" footer to every normal page.  (The first page of each chapter is not a "normal" page.)
\fancyfoot[C]{\emph{DRAFT Printed on \today}}  

% This highlights (in blue) corrections marked with (for words) \mccorrect{blah} or (for whole
% paragraphs) \begin{mccorrection} . . . \end{mccorrection}.  This can be useful for sending a PDF of
% your corrected thesis to your examiners for review.  Turn it off, and the blue disappears.
\correctionstrue


%%%%% BIBLIOGRAPHY SETUP
% Note that your bibliography will require some tweaking depending on your department, preferred format, etc.
% The options included below are just very basic "sciencey" and "humanitiesey" options to get started.
% If you've not used LaTeX before, I recommend reading a little about biblatex/biber and getting started with it.
% If you're already a LaTeX pro and are used to natbib or something, modify as necessary.
% Either way, you'll have to choose and configure an appropriate bibliography format...

% The science-type option: numerical in-text citation with references in order of appearance.
\usepackage[style=authoryear-ibid, sorting=nty, backend=biber, doi=false, isbn=false]{biblatex}
\newcommand*{\bibtitle}{References}
\usepackage{textgreek}

% The humanities-type option: author-year in-text citation with an alphabetical works cited.
%\usepackage[style=authoryear, sorting=nyt, backend=biber, maxcitenames=2, useprefix, doi=false, isbn=false]{biblatex}
%\newcommand*{\bibtitle}{Works Cited}

% This makes the bibliography left-aligned (not 'justified') and slightly smaller font.
\renewcommand*{\bibfont}{\raggedright\small}

% Change this to the name of your .bib file (usually exported from a citation manager like Zotero or EndNote).
\addbibresource{references.bib}


% Uncomment this if you want equation numbers per section (2.3.12), instead of per chapter (2.18):
%\numberwithin{equation}{subsection}



%%%%% THESIS / TITLE PAGE INFORMATION
% TODO: This well need to go to comply with my university
% Everybody needs to complete the following:
\title{Glial mechanisms contributing to neuronal network dysfunction in Alzheimer’s Disease mouse models}
\author{Josef Bitzenhofer}
% \college{Your College}

% Master's candidates who require the alternate title page (with candidate number and word count)
% must also un-comment and complete the following three lines:
%\masterssubmissiontrue
%\candidateno{933516}
%\wordcount{28,815}

% Uncomment the following line if your degree also includes exams (eg most masters):
%\renewcommand{\submittedtext}{Submitted in partial completion of the}
% Your full degree name.  (But remember that DPhils aren't "in" anything.  They're just DPhils.)
\degree{Doctor of Medicine (MD)}
% Term and year of submission, or date if your board requires (eg most masters)
% TODO: update this
\degreedate{2027}


%%%%% YOUR OWN PERSONAL MACROS
% This is a good place to dump your own LaTeX macros as they come up.

% To make text superscripts shortcuts
	\renewcommand{\th}{\textsuperscript{th}} % ex: I won 4\th place
	\newcommand{\nd}{\textsuperscript{nd}}
	\renewcommand{\st}{\textsuperscript{st}}
	\newcommand{\rd}{\textsuperscript{rd}}

%%%%% THE ACTUAL DOCUMENT STARTS HERE
\begin{document}



%%%%% CHOOSE YOUR LINE SPACING HERE
% This is the official option.  Use it for your submission copy and library copy:
% \setlength{\textbaselineskip}{22pt plus2pt}
% This is closer spacing (about 1.5-spaced) that you might prefer for your personal copies:
\setlength{\textbaselineskip}{18pt plus2pt minus1pt}

% You can set the spacing here for the roman-numbered pages (acknowledgements, table of contents, etc.)
\setlength{\frontmatterbaselineskip}{17pt plus1pt minus1pt}

% Leave this line alone; it gets things started for the real document.
\setlength{\baselineskip}{\textbaselineskip}


%%%%% CHOOSE YOUR SECTION NUMBERING DEPTH HERE
% You have two choices.  First, how far down are sections numbered?  (Below that, they're named but
% don't get numbers.)  Second, what level of section appears in the table of contents?  These don't have
% to match: you can have numbered sections that don't show up in the ToC, or unnumbered sections that
% do.  Throughout, 0 = chapter; 1 = section; 2 = subsection; 3 = subsubsection, 4 = paragraph...

% The level that gets a number:
\setcounter{secnumdepth}{2}
% The level that shows up in the ToC:
\setcounter{tocdepth}{2}


%%%%% ABSTRACT SEPARATE
% This is used to create the separate, one-page abstract that you are required to hand into the Exam
% Schools.  You can comment it out to generate a PDF for printing or whatnot.
% TODO: do we want this?
%\begin{abstractseparate}
%	\input{text/abstract} % Create an abstract.tex file in the 'text' folder for your abstract.
%\end{abstractseparate}


% JEM: Pages are roman numbered from here, though page numbers are invisible until ToC.  This is in
% keeping with most typesetting conventions.
\begin{romanpages}

% JEM: By default, this template uses the traditional Oxford "Belt Crest". Un-comment the following
% line to use the newer, "Blue Square" logo:
% \renewcommand{\crest}{{\includegraphics[width=4.2cm, height=4.2cm]{figures/newlogo.pdf}}}

% Title page is created here
% \maketitle

%%%%% DEDICATION -- If you'd like one, un-comment the following.
%\begin{dedication}
%This thesis is dedicated to\\
%someone\\
%for some special reason\\
%\end{dedication}

%%%%% ACKNOWLEDGEMENTS -- Nothing to do here except comment out if you don't want it.
\begin{acknowledgements}
 	\input{text/acknowledgements}
\end{acknowledgements}

%%%%% ABSTRACT -- Nothing to do here except comment out if you don't want it.
\begin{abstract}
	\input{text/abstract}
\end{abstract}

%%%%% MINI TABLES
% This lays the groundwork for per-chapter, mini tables of contents.  Comment the following line
% (and remove \minitoc from the chapter files) if you don't want this.  Un-comment either of the
% next two lines if you want a per-chapter list of figures or tables.
\dominitoc % include a mini table of contents
%\dominilof  % include a mini list of figures
%\dominilot  % include a mini list of tables

% This aligns the bottom of the text of each page.  It generally makes things look better.
\flushbottom

% This is where the whole-document ToC appears:
\tableofcontents

\listoffigures
	\mtcaddchapter
% \mtcaddchapter is needed when adding a non-chapter (but chapter-like) entity to avoid confusing minitoc

% Uncomment to generate a list of tables:
%\listoftables
%	\mtcaddchapter

%%%%% LIST OF ABBREVIATIONS
% This example includes a list of abbreviations.  Look at text/abbreviations.tex to see how that file is
% formatted.  The template can handle any kind of list though, so this might be a good place for a
% glossary, etc.
\include{text/abbreviations}

% The Roman pages, like the Roman Empire, must come to its inevitable close.
\end{romanpages}


%%%%% CHAPTERS
% Add or remove any chapters you'd like here, by file name (excluding '.tex'):
\flushbottom
\chapter{\label{ch:1-intro}Introduction} 

\minitoc

\section{Alzheimer's Disease}
\section{Mouse Models of AD: APP$^{NL-G-F}$}
% TODO: Something more introductary, e.g. about transgenic mice?
% TODO: then emphasising how it is better?
The APP$^{NL-G-F}$ mouse model is based on a C57BL/6 background carrying mutations of human familial AD patients, namely the Swedish (NL), Beyreuther/Iberian (F) and Arctic (G) mutations.
This model of Aβ amyloidosis notably does not show any overexpression of amyloid precursor protein (APP), but rather overproduction of A\textbeta$_{42}$. Thus, these mice exhibit Aβ amyloidosis, with plaque deposition in the cortex from as early as 2 months of age, showing near saturation at 7 months of age. Notably, there is considerable microgliosis and astrocytosis as hallmarks of neuroinflammation. Furthermore,  APP$^{NL-G-F}$ exhibit moderate memory impairments at 6 months of age, as measured by reduced alternation in the Y-Maze test. The age-dependent nature of pathology in this model is better suited to model the course of the disease in humans. \citep{Saito2014}
\section{Recording Activity of Neurons and Oligodendrocytes: In Vivo Two-Photon Calcium Imaging}
% TODO: some citation needed, more explanation about calcium indicators, two-photon effect
Neural activity can be detected using calcium indicators: During neuronal activity, extracellular calcium enters the cell.\\
Among calcium indicators, the subclass of genetically encoded calcium indicators (GECI) are especially frequently used. An optimal GECI exhibits high levels of expression in desired cells, high sensitivity (detection of action potentials), has fast kinetics (short rise time, short decay time), a high signal-to-noise ratio, a high tolerance to saturation, a high temporal resolution and a linear increase of fluorescence when spiking events coincide. Also, especially for recordings weeks after the injection, GECIs should maintain their fluorescence dynamics over time, should have little to no cytotoxic properties and should not suffer extensively from photobleaching. \cite{Zhang2023} \\
\textit{Zhang et al.} designed a family of calcium sensors, GCaMP8, a GECI based on green fluorescent protein (GFP). This calcium sensor is coupled to a synapsin-1 promotor, resulting in broad expression in neurons, excluding the nuclei. The entities in this family show different properties: GCaMP8f (\textit{fast}) exhibits especially fast rise and decay times, but lower sensitivity. GCaMP8s (\textit{sensitive}), on the other hand, has higher sensitivity, but slower kinetics as a trade-off. GCaMP8m (\textit{medium}) is a compromise with fast rise but medium decay times. \cite{Zhang2023} \\
This project required a calcium sensor that is especially sensitive and able to resolve single action potentials up to trains of spikes. Also, as the time scale of this project span several months, maintaining the same calcium dynamics over time was crucial. \\
For this project, we chose to use GCaMP8 because of its robust sensitivity and favourable dynamics. In addition, this virus allows for long-scale imaging projects as dynamics are almost unchanged over short periods of time, even weeks after the injection.  \cite{Zhang2023} We then decided to use the GCaMP8\textit{s} subtype because of its higher sensitivity and still more than robust temporal resolution. 
\section{Dysfunction of Oligodendrocytes in Alzheimer’s Disease}
% TODO: Sasmita et al.??
% TODO: add that oligodendrocytes produce even more Aβ?
Using single-nucleus RNA sequencing in human brain tissue of AD patients and healthy controls, previous work could show that oligodendrocytes express all the genes required to produce Aβ, including amyloid precursor protein (APP), beta secretase (BACE1) and components of gamma secretase such as presenilin 1 (PSEN1). Aβ production has been shown in oligodendrocytes derived from human familial AD patients' induced pluripotent stem cells (iPSCs). By inhibiting BACE1 through the administration of NB-360 in said oligodendrocytes, Aβ production could be significantly reduced to approximately 20\% of pre-treatment levels. An APP$^{NL-G-F}$ mouse model which allowed for conditional BACE1 knockout in either oligodendrocytes or neurons has been generated. When conditionally knocking out BACE1 in oligodendrocytes of 4 month old mice, the plaque load measured in number of plaques per mm$^2$ was reduced by 25\% compared to unmodified APP$^{NL-G-F}$ mice. Interestingly, in APP$^{NL-G-F}$ mice where BACE1 was knocked out in neurons, plaques were almost eliminated. Furthermore, neuronal hyperactivity in APP$^{NL-G-F}$ mice assessed by mean firing rate in hertz could be ameliorated by knocking out BACE1, both in neurons and oligodendrocytes. \citep{Rajani2024}
\break
Another study knocking out BACE1 in 6 month old mice with the same genetic background (APP$^{NL-G-F}$) in either oligodendrocytes or excitatory neurons further validated these findings as they reported a 30\% and 95-98\% reduction of plaque burden compared to controls, respectively. Interestingly, there was still noticeable plaque deposition in brains of ExN-BACE1 knockout mice at 12 months of age, further underlining the existence of non-neuronal sources of Aβ.\citep{Sasmita2024}
% TODO: there are more papers showing the latter argument!
% TODO: talk about 'sigmoidal growth curve', and threshold levels?
% TODO: Careful: the last sentence is prety close to sasmita et al.
Both studies highlight that plaque deposition depends on neuronal Aβ and the non-linear relation of APP processing, Aβ production and plaque formation. \citep{Sasmita2024, Rajani2024}
\include{text/ch2-litreview}
\chapter{Materials and Methods}
\label{ch:2-methods}

\minitoc

\section{Animals}
% Can I have what Rikesh wrote there? Careful, it is very similar
Tamoxifen (MP Biomedicals; 156738) was dissolved in corn oil at 25 mg/ml. All animals were treated with tamoxifen starting from 4-5 weeks of age for 4 weeks. Mice received 100 mg/kg tamoxifen intraperitoneally on weekdays.
% TODO: ask Jimmy or Rob, should I even report it was a "dirty" wildtype?
For the MBP-GCaMP injected mouse, we used an APP/PS1/rTg21221 mouse which was negative for all mutations. Additionally, this mouse has not been injected with tamoxifen.
\section{Surgical procedures}
Mice received 3 \% isofluorane for induction of general anaesthesia. Isofluorane levels were lowered to 1.5-2\% for maintenance throughout the surgery. Anaesthetic depth was assessed by monitoring the pedal withdrawal reflex and the breathing rate. Eye ointment was provided by applying Viscotears (TODO: add supplier) and body temperature maintained by placing the mouse on a heat pad. Carprofen was administered subcutaneously prior to the incision. Mice were then transferred to a Mouse Ultra Precise Stereotaxic Instrument (World Precision Instruments, WPI) and head fixed using ear bars. The scalp was shaven, disinfected using dilute chlorhexidine and then cleaned with ethanol.  EMLA-Cream (TODO: add supplier and stuff) containing Lidocain and Prilocain was applied to the scalp prior to incision of the skin. After incision of the skin, the skin margins were glued to the skull using Vetbond (3M).
A 3.5 mm wide craniotomy was drilled  with the centre being 2.0 mm posterior to Bregma to expose the retro-splenial cortex, using an OmniDrill35 (WPI).
pGP-AAV-syn-jGCaMP8s-WPRE (addgene; 162374-AAV1) was diluted 1:2 using PBS. 1 µl of the dilution was injected into each hemisphere, at 0.75 mm depth, at a rate of 100 nl per minute using a 10 µl Hamilton syringe, secured and controlled by a UMP3 UltraMicroPump (WPI).
Shortly after the injections, a cranial window was implanted and secured using Vetbond (3M). A customly designed titanium head plate was glued in place using superglue and then the incision was closed using dental cement (Super-Bond; Sun Medical), covering the skull, window and head plate.
Anesthesia was turned off and buprenorphine administered immediately afterwards. Postoperative pain relief was administered by providing carprofen in mice's drinking water for three days following the procedure. Animals were monitored for three postoperative days, assessing weight and health, the latter by using scoring sheets.
\subsection{MBP-GCaMP}
The methodology for the MBP-GCaMP injected mouse followed the same surgical procedure described above, with the exception that MBP:jGCaMP8s AAV5 was used (TODO: Sverre Grødem, University of Oslo).
\section{Water restriction}
Animals were allowed to recover from surgery for at least XX (TODO!) days before being put on water restriction. Mice received approximately 40 ml of water per kilogram bodyweight, i.e. 1 ml of water for a standard 25 g mouse. No animal was allowed to lose more than 20\% of the pre-water restriction weight and / or show pain, discomfort, or distress. The animals were weighed and monitored daily.
For the first behavioural time point, water restriction began two days before commencing the habituation scheme. For the second behavioural time point, mice were put on water restriction three days before starting habituation.
\section{Behavioural task}
We designed a spatial-navigation task (TODO: is this correct?) to investigate differences in learning, behaviour and neural activity patterns in the three groups mentioned above.
For this, we built a custom behavioural training rig using the International Brain Laboratory protocol (TODO: https://elifesciences.org/articles/63711, also read!) and adapted IBL rig code (https://github.com/int-brain-lab/iblrig, version 8).
We built two versions of our behavioural training rig, one within a box (TODO: wording) and another one under our two-photon microscope for simultaneous imaging.
Both contained a custom built linear treadmill, three LCD screens covering 240° (TODO: is this correct???, if so, citation?) of the mouse's view and an optimal lickometer (1020; Sanworks) or a custom built electrical conductive sensor for lick detection.
% TODO: This still needs work and referencing to Sun et al.
Adapted from \parencite{Sun2021}, mice were trained to learn the distinction of two textures, \textit{pebble} and \textit{black-and-white circles}. The animals were assigned a rewarded texture randomly.  Mice were head fixed to run on a linear treadmill and presented with corridors of either texture, alternated in a random order. Each passage through a corridor was considered a \textit{trial}. The mice had to move on the treadmill to move forward in the corridor. In the rewarded context, mice received a water reward after having moved 180 cm, no such reward was delivered in the unrewarded context. After this \textit{reward zone}, an \textit{inter-trial-interval (ITI)} of 20 seconds followed where the screens went black. Trials which were not completed after 5 minutes were considered as timed-out and the mouse was teleported to the beginning of another trial. Each day of training was considered a \textit{session}, and each session was limited to approximately 60 minutes.
Learning was measured by differences in speed and licking in both contexts: Naive mice show unspecific licking, especially in the reward zones, in both contexts and mostly similar speeds. Mice which have learned the task will pre-dominantly lick in the rewarded context, and especially so in the reward zone and in the area before, which we considered the \textit{anticipatory zone}.
We computed a learning metric following this formula:
\begin{equation}
\text{Learning metric} = \frac{d^{\prime}(\text{licking}) + d^{\prime}(\text{speed})}{2}
\end{equation}
(TODO: more explanation needed?)
After completing one or multiple sessions, animals which showed a learning metric above one for at least 50 trials were considered having learned the distinction and subsequently moved to the next stage.
\section{Behavioural paradigm}
\subsection{First behavioural time point}
Habituations started at an age of XX (TODO!), after starting the water restriction and lasted for 9 days. Briefly, mice were habituated to being handled and head fixed for progressively longer periods of time without showing any signs of pain, discomfort or distress. In addition, the habituation protocol included running on the linear treadmill, and licking a lick spout (TODO: is this word okay?). After 5-7 days of habituation in the training box (TODO: wording), we performed an expression check. Mice which exhibited sufficient viral expression for two-photon imaging were habituated on the rig under the two-photon microscope for the remaining days. Those mice, which did not show sufficient expression where moved back to the training box and kept as behaviour-only mice.
A subset of mice received an awake baseline recording (TODO!) while being head fixed and in a plastic tube rather than on the linear treadmill.
Training of animals started at an age of 4.5 to 5 months (TODO: verify!): First, mice were introduced to the task and both textures without receiving any rewards in any context (\textit{unsupervised learning}). Then, as mentioned above, each mouse was randomly assigned an \textit{initially rewarded texture}. After having learned the initial distinction, the rewarded texture for the individual mouse was swapped (\textit{reversal learning}). Once mice learned the reversed distinction, they were done with the first time point and taken off water restriction.
\subsection{Second behavioural time point}
At least 60 days after completion of the first behavioural time point, mice were tested for memory recall and learning flexibility (TODO: wording?).
Mice were re-habituated after being put on water restriction until they did not exhibit any signs of pain, distress or discomfort and until they reliable ran on the treadmill and licked the spout. This usually took 2-3 days of re-habituation (TODO: verify!). The viral expression was re-checked on the first day of re-habituation to decide whether the mice should be moved to the training box (TODO: wording!).
Re-training of animals started at an age of XX months (TODO: verify!). Mice started with the last reward condition they were exposed to \textit{memory recall}, i.e. the rewarded texture in the \textit{reversal learning}. Once mice learned the distinction, the rewarded texture was yet again swapped to the initially rewarded texture \textit{recall reversal learning}. After having re-learned the initial reward condition, mice were taken off water restriction.
\section{Two-photon recordings and data processing}
%TODO: where should I report the expression checks etc.???

%TODO: is this too close to the section in the Isaac's lab paper which Rob did?
\subsection{Image acquisition}
\label{Image acquisition}
For in-vivo two-photon calcium imaging, we used a custom-built resonant-scanning two-photon microscope (Independent NeuroScience Services), equipped with a tunable laser (Coherent Chameleon Discovery NX) and using a 16x water immersion objective (0.8 numerical aperture, Nikon). Image acquisition was controlled by ScanImage (MBF Bioscience) and performed at a wavelength of 910-920 nm and fluorescence detected with a GaAsP photomultiplier tube (Hamamatsu). The images were acquired at 30 fps and 2x zoom, with 512x512 pixels covering a FOV of 500x500 µm.
\subsubsection{Repeated recordings}
On the day of the first unsupervised learning session, a reference FOV was selected for each mouse to be imaged from in all subsequent recordings. The criteria used to determine the best FOV were the following:
\begin{enumerate}
    \item number of active cells (i.e., cells showing calcium dynamics)
    \item distance from probable injection site
    \item absence of larger blood vessels, absence of haemorrhages and petechiae
\end{enumerate}
%TODO: reference the nature communications paper which I presented in JC?
Then, for the subsequent days, we tried to image from the same FOV, indentifying it by:
\begin{enumerate}
    \item patterns of larger blood vessels
    \item patterns of smaller blood vessels
    \item depth from brain surface
    \item distinct, "landmark" cells
\end{enumerate}
% TODO: something about simultaneous recordings?
\subsubsection{Reactivation sessions}
For a subset of mice, we wanted to look at reactivation of neural activity patterns after the mouse has finished a behaviour session.
For this, we modified the behavioural task analogously to (TODO: Grosmark et al.!), so that we imaged from the animal for 15 mins, prior to the session, during the task, and then for another 15 mins after the session.
\subsubsection{MBP-GCaMP recordings}
The acquisition of recordings with the MBP-GCaMP injected mouse follows the procedure laid out in \ref{Image acquisition}, but with different zoom levels.
%TODO: mention all recordings or just the ones used in the "analysis"
At approximately 80 µm depth, cells from one FOV were imaged from at 1x and 2x zoom, for 10 mins each.
% TODO: was it the same FOV
For one recording, the laser power was alternated to check whether calcium dynamics were induced by photostimulation (TODO: is this the correct word?). Cells were imaged for 3 minutes each with 33, 52 and 74 mW power on sample.
\subsection{Calcium activity detection}
Raw images were processed in suite2p (https://github.com/MouseLand/suite2p) for motion correction and calcium signal extraction. Calcium signals were then deconvolved using an OASIS deconvolution algorithm\\ (https://github.com/j-friedrich/OASIS).
% TODO: probably also cite OASIS paper?
For analysis of behavioural data and aligning calcium signals or deconvolved spikes with behavioural events, we used custom-written Python scripts (Python 3.12.2).
%TODO: Did we change the suite2p settings for the MBP GCaMP?
%TODO: explain dff
For some analyses we computed \(\Delta \)F/F as follows:
Firstly, we subtracted neuropil signal from the fluorescence signal to account for neuropil contamination.
\begin{equation}
    F = F_{\text{raw}} - (F_{\text{neu}} \times 0.7)
    \label{eq:f}
\end{equation}
Then, we normalised fluorescence to mean fluorescence ($F_0$).
\begin{equation}
    \text{ΔF/F} = \frac{F(t) - F_0}{F_0}
    \label{eq:dFF}
\end{equation}
For the MBP-GCaMP mouse, the calcium activity detection followed the same procedure laid out above, with the exception that the no deconvolution built into suite2p was used. Regions of interests (ROI) were selected by identifying cells by their shape.
\subsection{Calcium activity in oligodendrocytes}
For plotting calcium traces of oligodendrocyte-shaped and neuron-shaped cells, cells were manually categorised into each category by cell size and shape (oligodendrocytes are smaller, neurons are bigger and more roundly shaped) (TODO: citation needed?). Dff of XX cells each were plotted.
For cell size comparisons, the suite2p built in output parameter "radius" of the ROI was used to compute the individual cell sizes. 
% TODO: alternating power recording??
\section{Tissue and blood biomarkers}
\subsection{Blood samples}
On average, 180 µl of blood were taken from the saphenous vein from each animal after completing the first and second behavioural time point. The blood was cooled and immediately spun down at 3,000 rpm for 10 minutes. Then, the plasma was taken up using a pipette and frozen.
%TODO: perhaps something about blood biomarkers in the future
\subsection{Brain extraction}
%TODO: wording? which PFA? should I mention coronal sections?
After mice finished the second behavioural time point, intracardiac perfusion of 1x PBS (Gibco) was performed, brains were extracted and the hemisphere which was imaged from was post-fixed in 4\% paraformaldehyde (PFA) for 24 hours. Afterwards, this hemisphere was cryopreserved through increasing concentrations of sucrose, embedded in optimal cutting temperature compound (OCT; CellPath) and frozen on dry ice. (TODO: is this too close to Rikesh's paper?). The other hemisphere was flash frozen for XX. (TODO: specify).
For behaviour-only mice, the decision of which hemisphere to use for sectioning or XX (TODO: specify), respectively, was made randomly.
\section{Immunohistochemistry}
\label{Immunohistochemistry}
For immunohistochemistry, cryopreserved brains were cut into 10 µm thick sections using a cryostat (CM1860 UV, Leica Microsystem) and frozen.
Samples were rinsed with PBS (Gibco) before blocking, using a blocking solution containing 10\% normal goat serum (TODO: supplier?) in 0.2\% Triton X in PBS for three hours at room temperature. Then, samples were incubated with primary antibodies in blocking solution overnight at 4°C.
After washing three times for 10 mins each with 0.2\% Triton X in PBS, samples were incubated with secondary antibodies in blocking solution for three hours at room temperature.
Following another two washing steps, DAPI was added to the third wash (1:10,000; TODO: supplier) to stain nuclei.
Lastly, samples were mounted using ProLong Glass Antifade Mountant (Invitrogen) and secured with CoverGrip Coverslip Sealant (Biotium).
\subsection{MBP-GCaMP}
% TODO: are the suppliers etc correct
For validation of sensitivity and specificity of the MBP-GCaMP virus, the injected mouse was sacrificed and perfused. Samples were obtained and stained following the procedure described above. We used rabbit-anti-ASPA (ABN1698; Sigma), mouse-anti-Olig2 (MABN50, Millipore) and chicken-anti-GFP (ab13970; Abcam) in blocking solution (all 1:500). For secondary antibodies, we used goat-anti-rabbit (647 nm; A-21245; Invitrogen), goat-anti-mouse (594 nm; A-11005; Invitrogen) and goat-anti-chicken (488 nm; A-11039; Invitrogen) antibodies in blocking solution (all 1:500).
To further investigate whether neurons, astrocytes or microglia had been labelled by the MBP-GCaMP virus, we developed two staining panels:
\begin{enumerate}
    \item (I) Primary antibodies: chicken-anti-NeuN (1:500, supplier?), mouse-anti-Olig2 (MABN50, Millipore), rabbit-anti-Iba1 (019-19741; Fujifilm Wako) in blocking solution (all 1:500)
    \item (II) Primary antibodies: chicken-anti-NeuN (1:500, supplier?), mouse-anti-Olig2 (MABN50, Millipore), rabbit-anti-GFAP (Z0334; Dako) in blocking solution (all 1:500).
\end{enumerate}
For secondary antibodies, we used goat-anti-chicken (488 nm; A-11039; Invitrogen), goat-anti-mouse (594 nm; A-11005; Invitrogen) and goat-anti-rabbit (647 nm; A-21245; Invitrogen) antibodies in blocking solution (all 1:500).
\end{enumerate}
\subsection{Myelination area fraction}
% TODO: wording
% TODO: check with Rikesh if this is consistent
% TODO: did she not wash after the secondaries???
For analysing myelination in prefrontal cortices (TODO: is this correct?), 2, 4 and 12 month old male and female APP$^{NL-G-F}$ mice sacrificed and perfused. Sections were obtained following the above mentioned procedure. Samples were first washed in PBS (Gibco), and then blocked using a blocking solution containing 10\% normal goat serum (TODO: supplier?) in 0.3\% Triton X in PBS for five hours at room temperature. Then, samples were incubated with primary antibodies in blocking solution overnight at 4°C. After washing three times for 10 mins each with 0.3\% Triton X in PBS, samples were incubated with secondary antibodies in blocking solution for two hours at room temperature. Subsequently, DAPI was diluted 1:5000 in PBS and put on the samples to stain nuclei.
Lastly, samples were mounted using ProLong Glass Antifade Mountant (Invitrogen) and secured with CoverGrip Coverslip Sealant (Biotium).
As primary antibodies, rabbit-anti-Olig2 (1:200; AB9610; Millipore), mouse-anti-CC1 (1:200; OP80; Millipore) and rat-anti-MBP (1:300; MCA409S; Bio-Rad) were used.
The secondary antibodies were goat-anti-rabbit (488 nm; A-11008; Invitrogen), goat-anti-mouse (594 nm; A-11005; Invitrogen) and goat-anti-rat (647 nm; A-21247; Invitrogen).
\section{Confocal imaging and analysis}
\subsection{MBP-GCaMP}
% TODO: do I feel confident to claim layer 2/3?
Confocal images were acquired using a confocal laser scanning microscope (LSM800; Zeiss).
For quantification, 11 FOVs were taken using a 40x oil immersion objective from the right-hand hemisphere on three different slices (3, 4 and 4 FOVs, respectively) and analysed using the Fiji build of ImageJ (version 2.16.0;\\ https://imagej.net/software/fiji/). Firstly, brightness and contrast were adjusted using the built-in function "Adjust Brightness/Contrast Auto", then positive cells were counted manually for each channel and all combinations of the individual channels using the "Cell Counter" function.
GFP-positive cells were considered labelled by the GCaMP virus. Cells were considered oligodendrocytes if they were positive for both ASPA and Olig2. Correctly labelled oligodendrocytes were defined by being positive for ASPA, Olig2 and GFP. Sensitivity and specificity were computed using custom-written Python scripts (Python 3.12.3) and Jupyter notebooks.
\subsection{Myelination area fraction}
% TODO: check if these informations apply to all images
% TODO: brain region?
% TODO: some had more or less steps in the z-stack, right?
% TODO: how much was actually analysed?
Confocal images were acquired using a confocal laser scanning microscope (LSM800; Zeiss) with a 40x oil immersion objective. Images were taken from the deep as well as superficial pre-frontal cortex (TODO: is this right?) and from the corpus callosum, with 5 FOVs per region on average. Each image encompasses a z-stack with a step size of 1 μm.\\
Further analysis was conducted using custom-written Python scripts (Python 3.12.3):
For each image, the slice with the highest intensity, along with the adjacent slice(s) (either top or bottom or both, depending on the initial slice's position) were included in the analysis. Then, area fraction was computed for each slice and the mean computed for the image.
% TODO: add the analysis steps from the python code
\section{Slide scanner imaging}
% TODO: add more imaging info
Fluorescence images were acquired using a microscope slider scanner (Axioscan 7; Zeiss) with a XXzoom objective. Images were taken from 8 slides with 3 sections each. For each section, the entire brain was captured.
% TODO: add more analysis steps
XX "analysis steps"


%% APPENDICES %% 
% Starts lettered appendices, adds a heading in table of contents, and adds a
%    page that just says "Appendices" to signal the end of your main text.
\startappendices
% Add or remove any appendices you'd like here:
\include{text/appendix-1}


%%%%% REFERENCES

% JEM: Quote for the top of references (just like a chapter quote if you're using them).  Comment to skip.
\begin{savequote}[8cm]
The first kind of intellectual and artistic personality belongs to the hedgehogs, the second to the foxes \dots
  \qauthor{--- Sir Isaiah Berlin \cite{berlin_hedgehog_2013}}
\end{savequote}

\setlength{\baselineskip}{0pt} % JEM: Single-space References

\renewcommand*\MakeUppercase[1]{#1}
\renewcommand{\cite}{\citep}
\bibliography{references}
% TODO: remove this eventually

\nocite{*}
\end{document}
